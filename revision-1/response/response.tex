% !TEX program: xelatex
\documentclass{article}
\usepackage[margin=1in]{geometry}
\usepackage[T1]{fontenc}
\usepackage{lmodern}
\usepackage[numbers]{natbib}
\usepackage{xcite}
\externalcitedocument{../revised-paper/methyl-valerate}
\usepackage{xr}
\externaldocument{../revised-paper/methyl-valerate}
\usepackage[capitalize]{cleveref}
\crefname{appendix}{}{}
\usepackage{siunitx}
\sisetup{
    separate-uncertainty=true,
    per-mode=symbol
}
\usepackage{url}

\newenvironment{reviewer}{\vspace{0.5\baselineskip}\begingroup\itshape\textbf{Reviewer:}}{\endgroup\vspace{0.5\baselineskip}}
\newenvironment{response}{\vspace{0.5\baselineskip}\textbf{Our Response:}}{\vspace{0.5\baselineskip}}
\newcommand{\review}[1]{{\itshape#1}}
\newcommand{\respond}[1]{\textbf{Our Response:} #1}

\begin{document}

\noindent Experiments and Modeling of the Autoignition of Methyl Pentanoate at Low to Intermediate Temperatures and Elevated Pressures in a Rapid Compression Machine \\
Ms.\ Ref.\ No.:  JFUE-D-17-02705

We would like to thank the reviewers for their thoughtful comments.
Below are detailed responses to each of the reviewers' comments.
In the attached marked revision, deleted text is marked in red and added text is marked in blue.

\vspace{\baselineskip}

\noindent \textbf{Reviewer \#1:}

\begin{reviewer}
%
    The authors study ignition of methyl pentanoate, a small methyl ester, in the current study. The
    data from the current study is helpful in establishing the reaction rates which can be
    systematically used for developing the kinetic models for large esters and hence very valuable
    for developing kinetic models of biodiesel. The study also shows the poor performance of the
    literature models in estimating the ignition delays underscoring the need for such novel
    experimental datasets and hence would recommend this work for publication. However, I think the
    manuscript needs to be revised before it can be published. Below are the specific comments which
    I think need to addressed before it can be published.
%
\end{reviewer}

\begin{response}
    Thank you for your comments. Detailed responses to each comment are below.
\end{response}

\begin{reviewer}
%
    Please improving the quality of figures. For eg: First stage and total ignition delay Symbols
    in Figs 5a \& 5b are very hard to differentiate. Figure 1 could also be improved.
%
\end{reviewer}

\begin{response}
    %
    We have increased the mark size on \cref{fig:simulation-comparison} to better distinguish the
    overall ignition delay and the first-stage ignition delay. On \cref{fig:ign-delay-def}, we made
    several changes to improve the clarity, including swapping the colors of reactive and
    non-reactive pressure traces, changing the non-reactive pressure to a dashed line style, and
    making the black lines thinner.
    %
\end{response}

\begin{reviewer}
    %
    Regarding the statement "The NTC region of MV is mapped out to provide additional information on
    the fidelity of using MV as a biodiesel surrogate." I am not convinced that MV could be a
    biodiesel surrogate, in my opinion it helps in understanding the chemistry of large esters.
    %
\end{reviewer}

\begin{response}
    %
    We agree with the reviewer that it is possible that MV will not be a good surrogate for
    biodiesel. Nonetheless, to our knowledge, there has been no study to suggest that it is not a
    good surrogate, probably because very few studies of MV have been conducted to date, so there is
    not enough information to judge. However, we think that judging the suitability (or lack
    thereof) for MV as a surrogate of biodiesel is outside the scope of this work, so we are simply
    noting that this study provides part of the necessary experimental information so that others
    may judge for themselves. We also agree with the reviewer that the scope of this work is
    actually wider than we had noted previously, so we have added a note that this study provides
    further information on the autoignition chemistry of large methyl esters near line 41:

    \begin{quote}
        %
        ``The NTC region of MV is mapped out to provide additional information on the fidelity of
        using MV as a biodiesel surrogate and insights into the autoignition chemistry of large
        methyl esters.''
        %
    \end{quote}
    %
\end{response}

\begin{reviewer}
    Specify the units of coefficients B, C  and, temperature T in equation (1)
\end{reviewer}

\begin{response}
    The coefficients are in \si{kPa} and \si{\kelvin}. This has been added to the text near line
    112 and in the caption of \cref{tab:antoine}.
    \begin{quote}
        ``... where \(A\), \(B\), and \(C\) are substance-specific coefficients, given in units of
        \si{\kelvin} and \si{\kPa}.''
    \end{quote}
\end{response}

\begin{reviewer}
    %
    The authors report an ignition delay of 100 ms at 640 K f=2, MV/air mixture at 15 bar. So at 30
    bar I would expect an ignition delay which is for sure will be more than 10 ms. So I am
    surprised by the statement "For the phi = 2:0 condition, only  PC = 15 bar is considered because
    we could not achieve TC values low enough that the ignition during the compression stroke can be
    prevented'. Can the authors comment on this.
    %
\end{reviewer}

\begin{response}
    %
    The reason for this is because of physical limitations of the experimental apparatus and the
    fuel. In particular, achieving sufficiently EOC low temperatures on our apparatus would require
    increasing the reaction chamber volume at the end of compression, which in turn requires
    substantially higher initial pressures to reach the EOC pressure of \SI{30}{\bar}. The initial
    pressure in the RCM reaction chamber is limited by the pressure in the mixing tank, and the
    total pressure in the mixing tank is limited by the vapor pressure of MV at the preheat
    temperature. The combination of these factors meant that we could not prepare a mixture with
    sufficiently high tank pressure to enable repeated experiments with the EOC volume required to
    reach EOC temperatures where the ignition delay would be measurable. However, we feel that this
    explanation would only confuse the manuscript, so we omit it.
    %
\end{response}

\begin{reviewer}
    %
    The comparison of ignition delays of MV from current work and that of Hadj-Ali et al .[9] would
    be helpful.
    %
\end{reviewer}

\begin{response}
    %
    The ignition delays of MV measured in the work of \citet{Hadj-Ali2009} were conducted at several
    pressures and roughly a single EOC temperature (near \SI{815}{\K}). However, none of the
    pressures considered in that study match the pressures considered in our study, so we feel that
    including the data on one of our existing plots would be very confusing. Moreover, the influence
    of facility-specific effects make a direct comparison of ignition delays from different RCMs
    somewhat challenging, particularly without experimental pressure traces of MV from the work of
    \citet{Hadj-Ali2009} to compare the facility effects. Therefore, we have not added a figure with
    this comparison.
    %
\end{response}

\begin{reviewer}
    %
    On discussion related to Dievart model in figure 5, it is surprising that phi=0.25 and phi=0.5
    simulations show NTC but phi=1 and phi=2 simulations do not. The trend is counter intuitive as
    the cool flame chemistry and NTC response are expected to increase with increase in equivalence
    ratio. This needs to be clarified. My guess is that the location of the NTC is not captured
    accurately by model. NTC region predicted by model is shifted to high temperatures where
    ignition is observed in RCM simulations. Showing Adiabatic constant volume simulations to
    understand the location of NTC at different equivalence ratios.
    %
\end{reviewer}

\begin{response}
    %
    We agree with the reviewer that the location of the NTC region with respect to inverse
    temperature is not well captured by the model. It does seem as though the NTC has been shifted
    to higher temperatures with respect to the data, in addition to the model being too reactive.
    This is supported by the newly added \cref{fig:conv-comparison}, which compares adiabatic,
    constant volume simulations with the RCM simulations from \cref{fig:simulation-comparison}. It
    is apparent from this figure that the predicted ignition delays in the $P_C = \SI{15}{\bar}$,
    $\phi = \num{2.0}$ and the $P_C = \SI{30}{\bar}$, $\phi = \num{1.0}$ simulations fall on the low
    temperature side of the NTC, while the $P_C = \SI{15}{\bar}$, $\phi =$ \numlist{0.25;0.50}
    simulated ignition delays are on the high-temperature side of the NTC, but approaching the NTC
    range such that they start to assume the characteristic curvature as the temperature decreases.
    In addition to the figure, we have added some paragraphs near line 293 to clarify this point,
    but we have not copied the added text here to save space.
    %
\end{response}

\begin{reviewer}
    %
    Figure 5 caption could also be revised.
    %
\end{reviewer}

\begin{response}
    %
    We have modified the text of the caption of \cref{fig:simulation-comparison} to read:

    \begin{quote}
        %
        ``Comparison of experimental (\(\tau\) and \(\tau_1\)) and simulated (\(\tau\)) ignition delays
        computed using the procedure described in \cref{sec:experimental-modeling}. a)
        \SI{15}{\bar}, b) \SI{30}{\bar}.''
        %
    \end{quote}
    %
\end{response}

\begin{reviewer}
    %
    From Figure 5a for phi=2, 15 bar and  fig. 5b phi=1, 30 bar the  Dievart et al. model
    consistently under predicts the ignition delays which contradicts the statement "While the
    model of Dievart et al. [15] over-predicts the first-stage ignition delay, it also over-predicts
    the first-stage pressure rise, thereby leading  to rapid overall ignition." This needs to be
    revised.
    %
\end{reviewer}

\begin{response}
    %
    This statement has been removed from the manuscript.
    %
\end{response}

\begin{reviewer}
    %
    On discussion related to RMG model from Figure 5, the authors state " model over-predicts the
    low-temperature ignition delays and does not predict.." define low temperature ignition delays."
    The ignition delays from RMG model and experiments cross over around 720 K so the statement that
    low temp. ignition delays are overpredicted is not completely accurate.
    %
\end{reviewer}

\begin{response}
    %
    In this context, we are defining low temperature ignition delays as those to the right of the
    experimental NTC region on the Arrhenius plot. We have added the following text near line 319:

    \begin{quote}
        %
        ``... the RMG model tends to over-predict the low-temperature overall ignition delays (i.e.,
        those to the right of the experimental NTC region on the Arrhenius plot)...''
        %
    \end{quote}
    %
\end{response}

\begin{reviewer}
    %
    Regarding the heat of formation shown in Table 3, it is interesting that the heat of formation
    of radical 3 and 4 estimated by RMG are different. I would request the authors check their
    numbers.
    %
\end{reviewer}

\begin{response}
    %
    We have checked the values for the estimates in the mechanism, and they are identical to the
    values in \cref{tab:heat-of-formation} and further identical to the values available from the
    RMG website at \url{http://rmg.mit.edu}. As such, no changes have been made to the manuscript in
    response to this comment.
    %
\end{response}

\begin{reviewer}
    %
    The authors use the branching ratios of fuel decomposition to fuel radicals to state the
    importance of fuel chemistry in the current work. The relative consumption shown in Table 4
    demonstrates that the branching ratios do not exhibit strong sensitive to thermochemistry which
    is supported by the fact that RMG and RMG switched show nearly identical branching ratios. I am
    not convinced that the importance of the thermochemistry is elucidated by the current discussion
    related to Table 4 and the discussion in my opinion does not provide any kinetic insights. May
    be the discussion could be removed.
    %
\end{reviewer}

\begin{response}
    %
    We have clarified the discussion of this point. In particular, we have emphasized that the
    results show that both the thermochemistry and reaction pathways are important, and cannot
    easily be disentangled from each other, requiring a thorough, detailed investigation of the
    methyl valerate reaction system which we consider outside the scope of the present work. In
    addition, we have noted that improper prediction of the thermochemistry of species may affect
    the RMG-generated model in particular due to the rate-based algorithm used in RMG, which may
    miss important reactions if the thermochemistry is not accurate. We have modified the text
    beginning near line 402, but do not copy the modification here for space.
    %
\end{response}

\begin{reviewer}
    %
    Alternatively, it would be interesting to compare the thermochemistry of the low temperature
    chemistry related reactions (RO2 chemistry) from models and ab-initio works like those Hayes and
    Burgess and also try adopting the reaction rates from ab-initio calculations and see to if the
    mechanism performance improves.
    %
\end{reviewer}

\begin{response}
    %
    We feel that the poor performance of both mechanisms indicates that a thorough and detailed
    study of the entire methyl valerate system is warranted, and small modifications, such as the
    replacement of only a few rates related to low-temperature chemistry, will not be sufficient to
    significantly improve the performance of the model. Moreover, we feel that such ad-hoc
    replacement of rates, even if the performance improves, could not be shown to be anything more
    than fortuitous without further detailed study, which is outside the scope of the present work.
    Finally, in a separate work, Di\'{e}vart et al. (Combustion and Flame, 159 (5) 1793--1805, 2012)
    note that ``preliminary tests with up-to-date kinetic parameters [including the work of Hayes
    and Burgess] for the generic low temperature scheme have shown an enhancement of the fuel
    reactivity at the end of the NTC region.'' Since the model already over-predicts the reactivity
    of the RCM data, particularly at the end of the NTC region, further enhancement is not likely to
    improve agreement.

    Therefore, we have not included these modifications in the manuscript; we have, however, added a
    note to the end of the discussion (near line 418) that references the work of \citet{Hayes2009}
    to indicate that some of the necessary detailed work has been started.
    %
\end{response}

\noindent \textbf{Reviewer \#2:}

\begin{reviewer}
    %
    The measurement of Methyl Valerate at these temperature and pressure conditions constitutes
    important work and the manuscript of well laid out and has potential to be widely circulated.
    The arguments in the article are sound and the figures are clear.
    %
\end{reviewer}

\begin{response}
    Thank you for your comments. A detailed response is below.
\end{response}

\begin{reviewer}
    %
    I would like to ask the authors to check one item definitively before the article is published.
    I am concerned that at these higher pressures and equivalence ratios close to 1, there would be
    ignition which occurs during the compression stroke. This would not be noticed if some of the
    fuel condenses during the charging process and thereby you are dealing with lean mixtures than
    those cited in the paper. The authors are encouraged to make sure this is not the case by
    possibly sampling the charge after filling through a GC or by a quantitative absorption method
    as that used routinely by Stanford.
    %
\end{reviewer}

\begin{response}
    %
    With regards to ignition during the compression stroke, we have compared each reactive pressure
    trace to its corresponding non-reactive trace and found they are in good agreement from the
    beginning to the end of compression, indicating that there is no heat release during the
    compression stroke. With regards to the filling process, we have thoroughly checked the
    temperatures throughout the apparatus and found them to be quite consistent, including the
    mixing tanks, the piping connections, and the reaction chamber itself. Moreover, we have
    conducted several studies in the past utilizing GC to determine mole fractions of components.
    All of these studies determined that the vaporization procedure used on this apparatus is
    sufficient to produce a homogeneous mixture of fuel and oxidizer, even for fuels with relatively
    low vapor pressures. We have added references to these prior studies in the present manuscript
    near line 110:

    \begin{quote}
        %
        ``Previous work has shown this procedure to completely vaporize the fuel and prevent fuel
        cracking during the heating process~\cite{Weber2011a,Kumar2009,Das2012}.''
        %
    \end{quote}
    %
\end{response}

\begin{reviewer}
    %
    In the future, a possible direct charging method might be encouraged instead of a mix vessel.
    %
\end{reviewer}

\begin{response}
    %
    We first note that direct charging methods have their own set of disadvantages, and are still
    required to address the issues of fuel condensation and fuel-oxidizer mixing. However, we feel
    that adding a detailed discussion of the advantages and disadvantages of each mixing method to
    the manuscript would be beyond the scope of the current work, and we have not made any changes
    in response to this comment.
    %
\end{response}

\begin{reviewer}
    %
    It would also be good to include some of the pressure profiles so that others in the community
    can check.
    %
\end{reviewer}

\begin{response}
    %
    Pressure profiles are available in the data files that will be posted on the Combustion
    Diagnostics Laboratory website and on FigShare, with URLs given in the manuscript. We have
    endeavored to provide the community with all of the information necessary to reproduce these
    experiments, including the pressure traces and versions of software packages used in the
    processing. Moreover, we include several pressure traces with discussion in
    \cref{sec:pressure-traces} of the manuscript. As such, no changes have been made to the
    manuscript in response to this comment.
    %
\end{response}

\noindent\textbf{Reviewer \#3:}

\begin{reviewer}
    %
    The authors perform a shock tube study measuring ignition delay times of methyl pentanoate, and
    compare with a kinetic model taken from the literature and one they generated using a Reaction
    Mechanism Generator, showing that neither model can predict the ignition data satisfactorily;
    they investigate and discuss possible reasons, but (very reasonably) leave solving this
    discrepancy as an open challenge to the combustion modeling community.
    %
\end{reviewer}

\begin{response}
    Thank you for your comments. Detailed responses to each comment are presented below.
\end{response}

\begin{reviewer}
    %
    I would suggest using the preferred IUPAC name "Methyl pentanoate" in the title (and mentioning
    the common name "Methyl valerate" in the abstract and a keyword).
    %
\end{reviewer}

\begin{response}
    %
    We have modified the title to read ``Experiments and Modeling of the Autoignition of Methyl
    Pentanoate at Low to Intermediate Temperatures and Elevated Pressures in a Rapid Compression
    Machine'', as the reviewer suggests. Methyl valerate and methyl pentanoate are already keywords.
    We have opted to retain methyl valerate as the primary name used for the compound throughout the
    text because of the convenient MV acronym (MP is ambiguous, because it could mean methyl
    propanoate as well).
    %
\end{response}

\begin{reviewer}
    %
    Regarding: > "available on the web at https://combdiaglab.engr.uconn.edu/database/rcm-database
    and on figshare at https://doi.org/10.6084/m9.figshare.5213341. In addition, ChemKED-format [33]
    files are available in the main ChemKED database repository at
    https://github.com/pr-omethe-us/ChemKED-database."

    None of the links work or contain the relevant data. Presumably the authors embargoed it and
    plan to post it upon acceptance of the manuscript.

    But I am pleased to see this sharing of data, as well as the extensive supplementary materials
    aiding in reproducible science. Bravo.
    %
\end{reviewer}

\begin{response}
    %
    This is indeed the case, and the relevant files will be posted after acceptance. We have double
    checked that the URLs are correct, just in case. Thank you for your kind words.
    %
\end{response}

\begin{reviewer}
    %
    Regarding: > "At 15 bar, the experimental ignition delays are under-predicted by the Di\'{e}vart et
    al. [15] model for the three equivalence ratios shown. For the <phi> = 0.25 and 0.5 conditions,
    the model appears to be predicting an NTC region of the ignition delays as the temperature
    decreases"

    Figure 5 does not show, to my eye, any NTC behaviour in the simulations, as asserted in this
    sentence. Perhaps if the simulations were plotted at a wider range of temperatures (they're
    simulations, so no experimental reason not to?) this may show up, but at present I feel the
    figure does not support this statement. For <phi>=0.5 simulations are not even plotted over the
    whole range of the experimental data, let alone extrapolated to lower temperatures where NTC may
    be predicted.
    %
\end{reviewer}

\begin{response}
    %
    We interpret the increasing curvature of the model response for the $P_C = \SI{15}{\bar}$, $\phi =$
    \numlist{0.25;0.5} conditions as the beginning of an NTC region. The RCM simulations are not
    extrapolated because they use the volume profile associated with a single experiment. However,
    the simulations are plotted over the whole range of the $\phi = \num{0.5}$ experiments; the
    large discrepancy between the simulations and experiments causes the illusion that the whole
    range is not covered.

    In addition, we have added a new \cref{fig:conv-comparison} that compares adiabatic, constant
    volume simulations with the model of \citet{Dievart2013} with the RCM simulations presented in
    \cref{fig:simulation-comparison}. The adiabatic, constant volume simulations can be extrapolated
    over a wide range of temperatures. From \cref{fig:conv-comparison}, it can be seen that the
    curvature in the RCM simulations at $P_C = \SI{15}{\bar}$, $\phi =$ \numlist{0.25;0.5} is
    related to the NTC region of the ignition delay, while the lack of curvature in the $P_C =
    \SI{15}{\bar}$, $\phi = \num{2.0}$ and the $P_C = \SI{30}{\bar}$, $\phi = \num{1.0}$ simulations
    is because those lie on the low-temperature side of the predicted NTC.
    %
\end{response}

\begin{reviewer}
    %
    I have other questions about possible causes of the discrepancies, but answering them is beyond
    the reasonable scope of this paper, and the fact that I have them is a good sign regarding the
    potential impact of this paper. I think people will find it interesting.
    %
\end{reviewer}

\begin{response}
    Thank you again for your comments. We are also interested in the resolution of these questions,
    but we agree that they are out of scope for this work.
\end{response}

\end{document}
